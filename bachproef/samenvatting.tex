%%=============================================================================
%% Samenvatting
%%=============================================================================

% TODO: De "abstract" of samenvatting is een kernachtige (~ 1 blz. voor een
% thesis) synthese van het document.
%
% Een goede abstract biedt een kernachtig antwoord op volgende vragen:
%
% 1. Waarover gaat de bachelorproef?
% 2. Waarom heb je er over geschreven?
% 3. Hoe heb je het onderzoek uitgevoerd?
% 4. Wat waren de resultaten? Wat blijkt uit je onderzoek?
% 5. Wat betekenen je resultaten? Wat is de relevantie voor het werkveld?
%
% Daarom bestaat een abstract uit volgende componenten:
%
% - inleiding + kaderen thema
% - probleemstelling
% - (centrale) onderzoeksvraag
% - onderzoeksdoelstelling
% - methodologie
% - resultaten (beperk tot de belangrijkste, relevant voor de onderzoeksvraag)
% - conclusies, aanbevelingen, beperkingen
%
% LET OP! Een samenvatting is GEEN voorwoord!

%%---------- Nederlandse samenvatting -----------------------------------------
%
% TODO: Als je je bachelorproef in het Engels schrijft, moet je eerst een
% Nederlandse samenvatting invoegen. Haal daarvoor onderstaande code uit
% commentaar.
% Wie zijn bachelorproef in het Nederlands schrijft, kan dit negeren, de inhoud
% wordt niet in het document ingevoegd.

\IfLanguageName{english}{%
\selectlanguage{dutch}
\chapter*{Samenvatting}
\lipsum[1-4]
\selectlanguage{english}
}{}

%%---------- Samenvatting -----------------------------------------------------
% De samenvatting in de hoofdtaal van het document

\chapter*{\IfLanguageName{dutch}{Samenvatting}{Abstract}}

Gezien het actuele belang van snelle en professionele softwareontwikkeling is er een nood aan tools die het ontwikkelingsproces nog kunnen versnellen en vereenvoudigen. In deze paper wordt er een onderzoek gedaan naar de \textit{Swift OpenAPI Generator}. Er wordt een antwoord gezocht op de vraag: “In hoeverre is de \textit{Swift OpenAPI Generator} in staat om snel een werkend prototype te genereren dat zowel geschikt is voor demonstratie- doeleinden als kan evolueren tot een productieklare back-end zonder dat een herimplementatie nodig is”.
\\ \\
Deze bachelorproef onderzoekt de efficiëntie en effectiviteit van de  \textit{Swift OpenAPI Generator} in het softwareontwikkelingsproces. Het doel is om er achter te komen of de tool een werkend prototype kan produceren, die later kan worden omgezet naar een volledige operationele back-end zonder dat er een volledige herimplementatie nodig is. Er zal gekeken worden naar het potentieel van de tool om de softwareontwikkeling te versnellen en de complexiteit te verminderen. Er zal gefocust worden op hergebruik van code en snelle iteraties. Dit kan zorgen voor waardevolle inzichten voor ontwikkelaars. 
\\ \\
Het plan van aanpak bestaat uit verschillende fases. In de eerste fase wordt er een diepgaand onderzoek verricht naar relevante literatuur en worden bestaande OpenAPI generators onderzocht. Daarna worden doelstellingen bepaald voor het verder verloop van het onderzoek. Hierna zullen er verschillende scenario’s gecreëerd worden met steeds moeilijkere API’s en zal er een \textit{Swift UI} gemaakt worden om deze later te testen. 
Vervolgens zullen alle mogelijke scenario’s worden uitgewerkt en getest. In een laatste fase zullen de ervaringen en uitkomsten geanalyseerd worden om inzicht te krijgen in de prestaties van de \textit{Swift OpenAPI Generator}. Ten slotte zal een conclusie uitgeschreven worden. 
\\ \\
Het onderzoek toont aan dat de \textit{Swift OpenAPI Generator} in staat is om snel een werkend prototype te genereren dat kan evolueren tot een productieklare back-end. Het integreren van een database is relatief eenvoudig en er kunnen aanpassingen gedaan worden aan het OpenAPI-document tijdens de ontwikkeling, wat de ontwikkelaars toe laat om te experimenteren tijdens het ontwikkelen. 
Back-end logica zoals \textit{logging}, \textit{metrics} en validatie kunnen worden toegevoegd, hoewel validatie arbeidsintensief is en nog verbetering vraagt. Authenticatie vormt nog een uitdaging. De gegenereerde back-end kan naadloos worden geïntegreerd in een \textit{client} applicatie.
\\ \\
De \textit{Swift OpenAPI Generator} is geschikt voor het maken van een back-end, maar wordt complexer bij geavanceerde functies zoals authenticatie. Het is nuttig voor onderwijsdoeleinden, waar studenten gemakkelijk een back-end kunnen opzetten zonder diepgaande kennis. Voor een complexere back-end zijn echter aanpassingen nodig aan de generator.

