%%=============================================================================
%% Conclusie
%%=============================================================================

\chapter{Conclusie}%
\label{ch:conclusie}

In dit onderzoek wordt er een antwoord gegeven op de onderzoeksvraag : “In hoeverre is de \textit{Swift OpenAPI Generator} in staat om snel een werkend prototype te genereren dat zowel geschikt is voor demonstratie- doeleinden als kan evolueren tot een productieklare back-end zonder dat een herimplementatie nodig is ”. Om een antwoord te geven op deze vraag werd de \textit{Swift OpenAPI} generator aan de hand van scenario’s uitgebreid getest. 

Uit dit onderzoek komen diverse bevindingen naar voren. Ten eerste wordt duidelijk dat het integreren van een database op een relatief eenvoudige en doeltreffende manier kan worden gerealiseerd. Dit werd gerealiseerd door \textit{migrations} en models toe te voegen aan het project. Gedurende het volledige ontwikkelproces van de back-end is het mogelijk om flexibel aanpassingen door te voeren aan het OpenAPI-document. Dit geeft ontwikkelaars de ruimte om te experimenteren en aanpassingen te maken waar nodig. 

Daarnaast is het mogelijk om diverse facetten van backend logica toe te voegen. Deze Facetten omvatten belangrijke elementen zoals \textit{logging}, \textit{metrics} en validatie. Interssant is dat \textit{logging} zowel op de standaardmethode als via een meer complexe \textit{middleware} kan worden geïmplementeerd. \textit{Metrics} kunnen eveneens worden toegevoegd met behulp van een \textit{middleware}. Validatie kan op een iets minder efficiënte manier worden toegevoegd. 
Om validatie toe te voegen werd aanvankelijk geprobeerd om models te gebruiken voor het valideren van gegevens, maar al snel bleek dit niet haalbaar vanwege de discrepanties tussen de gebruikte validatiemethoden en de \textit{Swift OpenAPI Generator}. Een alternatieve benadering, gebaseerd op het format zoals gespecificeerd in het OpenAPI-document, leek veelbelovend maar werkte niet goed met de generator. 
Uiteindelijk werd gekozen voor een pragmatische aanpak waarbij handmatige validatie werd uitgevoerd in de \textit{handler} voordat gegevens werden opgeslagen. Hoewel dit effectief bleek te zijn en voldeed het aan de validatiebehoeften, is dit zeer arbeidsintensief en zou het handiger zijn als je je de validatie zou kunnen toevoegen aan het OpenAPI document. 

Een andere cruciale bevinding die uit het gedetailleerde onderzoek naar voren kwam, betreft de kwestie van authenticatie. Hier zijn nog enkele uitdagingen die aandacht vereisen en waarvoor verdere verbeteringen en innovaties absoluut noodzakelijk zijn om de efficiëntie en veiligheid van het systeem te waarborgen.

Tot slot, maar zeker niet onbelangrijk, is het vermeldenswaard dat de backend, die zorgvuldig is ontworpen en gemaakt met behulp van de \textit{Swift OpenAPI Generator}, naadloos kan worden geïntegreerd in een \textit{client}applicatie. Dit is een belangrijke stap die kan worden bereikt door simpelweg de benodigde \textit{dependencies} toe te voegen en een geschikte netwerklaag te implementeren. Deze belangrijke bevindingen bieden waardevolle inzichten en vormen een solide basis voor verdere ontwikkeling en optimalisatie van de applicatiearchitectuur in de toekomst.

Hoewel de \textit{Swift OpenAPI Generator} over het algemeen geschikt is voor het genereren van een back-end, wordt de complexiteit ervan duidelijk wanneer men meer geavanceerde functies zoals authenticatie wil toevoegen. In mijn ervaring is de \textit{Swift OpenAPI Generator} een waardevol hulpmiddel voor educatieve doeleinden, waarbij het studenten in staat stelt om gemakkelijk een backend op te stellen, zonder echte back-end kennis. Echter, bij het ontwikkelen van een complexere backend, zijn aanpassingen aan de generator noodzakelijk om aan de vereisten te voldoen.