%%=============================================================================
%% Methodologie
%%=============================================================================

\chapter
{\IfLanguageName
    {dutch}
    {Methodologie}{Methodology}}%
\label{ch:methodologie}

Zoals bij de start van de meeste onderzoeken is het belangrijk om een grondig plan van aanpak op 
te stellen, dat de basis vormt voor het verkennen en evalueren van de \textit{Swift OpenAPI Generator}. 
Dit plan van aanpak is zorgvuldig gestructureerd en opgedeeld in verschillende fases, waarin elk aspect van de generator grondig wordt onderzocht en geanalyseerd. 
Het zorgt voor een duidelijk overzicht van de te volgen stappen en doelstellingen, om de functionaliteiten, mogelijkheden en toepassingen van de generator te begrijpen en te beoordelen. 
Het doorlopen van deze fases zorgt voor het verkrijgen van een diepgaand inzicht in de potentie en de bruikbaarheid van deze tool in diverse contexten en projecten. 

\section{Literatuurstudie}
In de eerste fase van dit onderzoek is het doel om allerhande informatie te verwerken om een diepgaande en grondige kennis te hebben over het onderwerp. Deze kennis opbouw wordt verder onderverdeeld in deelfases. Eerst wordt er onderzoek gedaan naar de OpenAPI-Specificaties. Zoals de naam van de te onderzoeken generator al laat verklappen, is de \textit{Swift OpenAPI Generator} zeer afhankelijk van deze specificaties en hun correcte toepassing. Het doel van deze fase is om een beter inzicht te krijgen in hoe OpenAPI in verschillende contexten wordt gebruikt en om te begrijpen hoe de specificaties generiek toepasbaar zijn. Naast het begrijpen van de OpenAPI-specificatie zal de tweede fase zich focussen op het verkennen van best practices en huidige trends in API-ontwikkeling.  \\ \\ Het is belangrijk om te begrijpen hoe back-ends doorgaans worden ontwikkeld en geoptimaliseerd, en welke methoden en technieken momenteel de voorkeur genieten in de industrie. In de derde en laatste deelfase wordt er nadruk gelegd op het begrijpen en leren gebruiken van \textit{Swift} in back-end ontwikkeling. In deze fase wordt onderzocht hoe \textit{Swift} kan worden ingezet om efficiënte en effectieve back-ends te bouwen, en in hoeverre dat deze taal gebruikt kan worden om aan de specifieke behoeften van verschillende projecten te voldoen. 

\section{Doelstellingen}
Na het voltooien van de vorige fase zullen er duidelijk doelstellingen worden vastgesteld voor de voortgang van het onderzoek. Deze doelstellingen, die zullen worden vastgelegd op basis van een grondige literatuurstudie, zullen ons in staat stellen om een beeld te vormen over wat er precies van de \textit{Swift OpenAPI Generator} wordt verwacht. Onder de belangrijkste capaciteiten die deze tool zou moeten hebben, bevindt zich de mogelijkheid om authenticatie- of validatieprocedures toe te passen. Daarnaast zou de \textit{Swift OpenAPI Generator} ook in staat moeten zijn om API-aanroepen te beperken tot een specifieke rol, waardoor de veiligheid en betrouwbaarheid van de processen die met deze tool worden uitgevoerd, kunnen worden gewaarborgd.

\section{Voorbereiding}
In de derde fase van het onderzoek zal men zich richten op de voorbereidingen van de vierde fase. Hier zal men zich concentreren op het creëren van een reeks mogelijke scenario's om de Swift OpenAPI-generator grondig te testen. Hierbij zal gestart worden met een heel simpele api, waarbij bij elke iteratie een iets complexere api vereist is. De OpenAPI-specificaties en de resultaten van de eerder uitgevoerde literatuurstudie zullen als leidraad dienen om deze potentiële scenario's te ontwerpen en te concretiseren. Nadat deze scenario’s zorgvuldig zijn uitgeschreven en gedocumenteerd, kan de overgang worden gemaakt naar de volgende fase. 

\section{Technisch aspect}
In deze fase van het project zullen de scenario's zorgvuldig worden uitgewerkt met de \textit{Swift OpenAPI Generator}. Het proces begint met het opstellen van het OpenAPI Document. Dit document zal de basis vormen voor het verdere ontwikkelingsproces. Daarna zal de \textit{handler} worden uitgewerkt, deze uitwerking zal steeds dieper en gedetailleerder worden naarmate het project vordert.
Nadat deze scenario's zorgvuldig zijn uitgewerkt en getest, kan de overgang worden gemaakt naar het testen van het \textit{client} gedeelte van de\textit{Swift OpenAPI Generator}. Dit gedeelte van de generator is van cruciaal belang, omdat het de manier is waarop gebruikers met de API zullen omgaan. Het \textit{client} gedeelte zal zo ontworpen worden dat gebruikers gemakkelijk kunnen observeren en begrijpen hoe de API functioneert.
Het uiteindelijke doel van het ontwikkelen van deze \textit{client} is niet alleen om een functio-neel product te creëren, maar ook om te onderzoeken of het mogelijk is om een snelwerkend en efficiënt prototype te creëren. 

\section{Conclusie}
Als laatste fase van dit onderzoek, zal er een grondige conclusie worden getrokken. Deze conclusie zal volledig gebaseerd zijn op de resultaten die verkregen zijn tijdens het onderzoek. Het hoofddoel van deze conclusie is om op een duidelijke en nauwkeurige manier een antwoord te formuleren op onze onderzoeksvraag. Daarnaast zullen we op basis van onze bevindingen en de conclusie die we hebben getrokken, een aanbeveling schrijven. Deze aanbeveling zal specifiek gericht zijn op het gebruik van de \textit{Swift OpenAPI Generator}. Hopelijk zullen deze bevindingen kunnen bijdragen aan
de verbetering en optimalisatie van het gebruik van deze tool in toekomstige projecten.


